\documentclass{article}[12]


\begin{document}

\section{Introduction}
What is physics?

The big questions of physics are the big questions of the universe:
what is it?  Why is it there?  Did it have to be this way, or could it
have been different?  Physics uses mathematics, experiment, and
explanation to come to terms with things that are almost unimaginable
in scale.  The Large Hadron Collider in Switzerland is smashing
particles called protons into each other at speeds very, very close to
the speed of light, and watching very closely to see all the debris
that come out of the collisions.  They are a little bit different
every time, due, apparently, to some randomness in the ether.

What do they do, when they track these debris?  They trace backward,
finding that the debris they see at the detectors (which, obviously,
can't be quite at the location of the collisions), have moved in
straight lines outward from previous collisions -- that themselves
traced backward into the singular moment of the original collision.
They calculate the momenta of these particles, and make calculations
about what kinds of interactions could have caused those particular
particles, with those particular momenta and directions to have been
produced. And every once in a while, they find that they cannot
explain the traces they see using any of the known particles, or any
of the known ways that those particles can interact with each other.
And if they see one of those strange events enough times (so they are
sure it wasn't a fluke of a mishap with the detectors), and they can
find someone who can suggest an explanation, they propose: ``HEY!!
We've found a BRAND NEW particle!! No one has EVER seen this before.
We are BAD-ASS!!''

The stakes are very high.  Billions of dollars.  The careers of
thousands of scientists and engineers.  But at heart, what they are
doing, is not so different from a traffic engineer -- surveying the
scene of an accident, photographing the final locations of the
vehicles, the specific damage done to the bodies, the tracks on the
pavement, to determine who was speeding or which vehicle crossed the
center-line.  Except that sometimes the answer is: ``A Higgs particle
suddenly came into existence and pushed Car #2 into the path of Car
#1.''

Physics I teaches you the basic principles, techniques, and thought
processes necessary to understand a crash scene or a particle trace
from the LHC.  Along the way, you learn how to analyze building
components for stability, the claims of ``green'' energy purveyors,
and the motions of spinning things.  You'll also practice developing
explanatory models for different phenomena, and testing those models
using experiments you will design yourself.  

Many people dislike science and math because it makes them feel
stupid.  Because scientists are unusually good at saying ``You're
wrong!''




* Units are things we measure with.  Some units are basic ( you can
measure them directly), others are derived (you get them by combining
other measurements).  You can get ``meters'' that will read out in all
kinds of units, but as much as possible, at least in this class, we
want to stick to the basic, fundamental units.  They are also called
SI units.  I swear to you, that the reason I want you to use them is
because they are \emph{easier} than the American or Imperial units.  

* That said, how do we measure \emph{speed}?  What are it's units?  Is
that a derived unit, or a basic one?

So one of the most common units of measure, that we see every day (if
we drive), is actually not in the SI system, and it's not
fundamental.  What are the fundamental units, of which the derived
unit of speed is composed?

\begin{tabular}{c|c|c|c|c|c}
50 miles & 5280 feet  & 12 inches & 2.54 centimeters & 1 meter & 1 hour \\
\hline
1 hour & 1 mile & 1 foot & 1 inch & 100 cm & 3600 seconds

\end{tabular}


\end{document}